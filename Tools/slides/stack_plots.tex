\documentclass[10pt,xcolor=svgnames,fleqn,aspectratio=169]{beamer}

\usepackage{tikz}
\usepackage{hyperref}
\usepackage{xcolor,colortbl}
\usepackage[absolute,overlay]{textpos}
\usepackage{graphicx}
\usepackage{cancel}

\def\blue{\color{DodgerBlue}}
\def\green{\color{Green}}
\def\red{\color{Red}}
\def\Orchid{\color{Orchid}}
\definecolor{DodgerBlueDark}{HTML}{1873cc}

\mode<presentation>
\usetheme{Copenhagen}
\usecolortheme[named=DodgerBlueDark]{structure}
%\useoutertheme{infolines}
\useoutertheme[footline=authortitle]{miniframes}
%\setbeamertemplate{headline}[infolines theme]
\setbeamertemplate{headline}[default]
\setbeamertemplate{navigation symbols}{}
\setbeamertemplate{footline}[miniframes theme]
\setbeamertemplate{footline}[page number]
\setbeamertemplate{itemize items}[triangle]

\newcommand{\dm}{$\Delta m$~}
\newcommand{\dphi}{$\Delta \phi$~}
\newcommand{\pt}{$p_{T}$~}
\newcommand{\texeta}{$\eta$~}

\title{Z to Invisible: Data and MC Stack Plots}

\author{\textcolor{DodgerBlueDark}{\bf C.~Smith\inst{1}}}
\institute{\inst{1} Baylor}
\date{\today}

\renewcommand{\arraystretch}{1.2}
\begin{document}

\begin{frame}[plain]
\maketitle
\end{frame}

\begin{frame}{Summary}
Data vs MC Comparisons
\begin{itemize}
\item Electron, Muon, and Photon Control Regions
\item Low \dm and High \dm selections
\begin{itemize}
\item Z mass cut applied in all lepton plots except for Z mass plot
\item High \dm mid \dphi selection for validation region applied in Z mass plots (which will be removed in future iterations)
\end{itemize}
\item Jet $p_T > 30$ GeV requirement for calculating all variables including b-jet variables
\begin{itemize}
\item Jet $p_T > 20$ GeV requirement will be used for b-jet variables in future iterations
\end{itemize}
\item Photon selection is still only \pt, \texeta, and loose ID
\begin{itemize}
\item Photon selection of direct, fragmentation, and fake photons will be used in future iterations
\item Could increasing the photon $p_{T} > 220$ GeV cut improve Data vs MC agreement?
\end{itemize}
\end{itemize}
\end{frame}

%%%%%%%%%%%%%%%%%%%%%%%%%%%%%%%%%%%%%%%%%%

\input{stack_snippet.tex}

%%%%%%%%%%%%%%%%%%%%%%%%%%%%%%%%%%%%%%%%%%

\end{document}
